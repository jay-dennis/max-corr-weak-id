%To Print Notes too
%\documentclass[11pt, notes]{beamer}

%To Compile Only slides
\documentclass[11pt]{beamer}

\usepackage{amsmath,amsthm,enumerate,graphicx,bm,type1cm,amssymb,epsfig,lscape,setspace,url,listings,color,float}
%wrapfig,tikz,calc}
\usepackage{fancybox,etoolbox}
\usepackage{hyperref}
\usepackage[round]{natbib}
%\usepackage{soul}

%\setstcolor{red}

\usepackage[normalem]{ulem}
\usepackage{xcolor}


\newcommand\redsout{\bgroup\markoverwith{\textcolor{red}{\rule[0.5ex]{2pt}{0.4pt}}}\ULon}


% % %%FOR PRINTING
%\usepackage{pgfpages}
%\pgfpagesuselayout{2 on 1}[letterpaper,border shrink=5mm]
% % %%\pgfpagesuselayout{resize to}[letterpaper,border shrink=5mm,landscape]
%


%\setlength{\oddsidemargin}{5mm}
%\setlength{\textwidth}{16cm}
%\setlength{\topmargin}{0pt}
%\setlength{\headheight}{0pt}
%\setlength{\headsep}{0pt}
%\setlength{\textheight}{23cm}
%\onehalfspacing
%\setbeamersize{text margin left = 10pt}




\usetheme{Madrid}
\setbeamersize{text margin left=1em}
\beamertemplatenavigationsymbolsempty

%\usetheme{Frankfurt}
%\useoutertheme{smoothtree}
%\useinnertheme{circles}


%\sectionfont{\large}

%\theoremstyle{definition}
%\newtheorem{mc}{MC}
%\newtheorem{thm}{Theorem}[section]
%\newtheorem{example}{Example}[section]
%\newtheorem{prop}{Proposition}[section]
%\newtheorem{pr}{Proof}
%\newtheorem{assump}{Assumption}[section]
%\newtheorem{lemma}{Lemma}[section]
%\newtheorem{definition}{Definition}[section]
%\newtheorem*{dfn}{Definition}
%\newtheorem*{rem}{Remark}
% \def\stackunder#1#2{\mathrel{\mathop{#2}\limits_{#1}}}
% \renewcommand{\theequation}{\thesection.\arabic{equation}}
%\newtheorem{corol}{Corollary}[section]

\undef{\lemma}
\newtheorem{lemma}{Lemma}

\undef{\assumption}
\newtheorem{assumption}{Assumption}

\undef{\remark}
\newtheorem{remark}{Remark}

\undef{\corollary}
\newtheorem{corollary}{Corollary}

\undef{\example}
\newtheorem{example}{Example}

\newtheorem{model}{Model}
  

\setlength{\bibsep}{0pt plus 0.3ex}

\allowdisplaybreaks

\restylefloat{table}

\newcommand{\beginbackup}{
   \newcounter{framenumbervorappendix}
   \setcounter{framenumbervorappendix}{\value{framenumber}}
}
\newcommand{\backupend}{
   \addtocounter{framenumbervorappendix}{-\value{framenumber}}
   \addtocounter{framenumber}{\value{framenumbervorappendix}} 
}

